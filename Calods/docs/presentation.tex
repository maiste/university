\documentclass[11pt]{beamer}
\usetheme{CambridgeUS}


\usepackage[utf8]{inputenc}
\usepackage{amsmath}
\usepackage{amsfonts}
\usepackage{amssymb}
\usepackage{minted}
\usepackage[french]{babel}


\author{Chaboche - Marais}
\title{CALODS - Concise Algorithm Language for Observing Distributed Systems}
%\setbeamercovered{transparent} 
%\setbeamertemplate{navigation symbols}{} 
%\logo{} 
%\institute{} 
%\date{} 
%\subject{} 
\begin{document}

\begin{frame}
\titlepage
\end{frame}

\begin{frame}
\tableofcontents
\end{frame}

\section{Introduction}
\begin{frame}{Introduction}{Notre objectif}
Projet de recherche avec M. Sangnier et M. Laroussinie, \textit{CALODS}:
\begin{itemize}
  \item définir un langage pour décrire les algorithmes distribués
  \item représenter graphiquement toutes les exécutions d'un programme
  \item créer un système de vérification de propriétés
  \item permettre d'ordonner les processus afin de contraindre l'exécution des programmes
\end{itemize}
\end{frame}

\begin{frame}{Introduction}{Notre projet}
Que fait \textit{CALODS} ?
\begin{itemize}
  \item production d'un fichier Graphviz représentant le diagramme d'états
  \item production d'un fichier \textit{Prism} correspondant à la traduction d'un programme \textit{CALODS}
  \item production d'un automate de Büchi non-déterministe pour ordonner l'exécution des programmes
\end{itemize}
\end{frame}

\section{Architecture, conception et gestion de projet}
\begin{frame}{Architecture}
Composition du projet :
\begin{itemize}
  \item compiler: % parser, module, gestion d'erreur, fifo
  \begin{itemize}
    \item But: générer une structure abstraite représentant le programme
    \item Dépendance: menhir
  \end{itemize}
  \item prism: % transformation d'ast, langage prism, base sur les expressions omega régulières
  \begin{itemize}
    \item But: traduire la structure abstraite en un programme \textit{Prism}
    \item Dépendance: Prism Model Checker
  \end{itemize}
  \item graph: % graphes, execution symbolique, Graphviz
  \begin{itemize}
    \item But: représenter le programme sous forme de diagramme d'états
    \item Dépendance: Graphviz
  \end{itemize}
\end{itemize}
\end{frame}

\begin{frame}{Conception}
Éléments de conception du projet:
\begin{itemize}
  \item \textbf{Modularité}: Compiler est une interface qui permet d'abstraire le code notre compilateur
  \item \textbf{Inter-dépendance}: Prism et Graph dépendent de Compiler pour la structure de données $\Rightarrow$ pas de dépendance entre Prism et Graph
  \item \textbf{Avantage}: La structure en modules permet de changer le contenu ou d'ajouter de nouveaux éléments facilement
  \item \textbf{Difficulté}: comprendre \textit{Prism} et générer les graphes en utilisant les bonnes structures de données
\end{itemize}
\end{frame}

\begin{frame}{Gestion de projet}
Répartition des rôles et organisation:
\begin{itemize}
  \item \textbf{Rôles}:
  \begin{itemize}
    \item Compiler: ensemble
    \item Prism: Valentin, rejoint par Étienne (module plus conséquent)
    \item Graph: Étienne
  \end{itemize}
  \item \textbf{Organisation}: réunions bi-mensuelles avec M. Sangnier et M. Laroussinie et échanges par mails
  \item \textbf{Git}: utilisation de branches pour éviter les conflits et d'issues pour les bugs
  \item \textbf{Tests}: présentation aux enseignants et exécution sur des exemples avec des résultats connus
\end{itemize}
\end{frame}

\section{Programmation}
\begin{frame}[fragile]{Module de compilation}
\begin{minted}[fontsize=\tiny]{ocaml}
(* Implement the compiler *)
module Calods  : Compiler with type ast = Ast.program = struct
  type ast = Ast.t

  let parse_filename file =
    let input = open_in file in
    let lexbuf = Lexing.from_channel input in
    let (header, processes) =
      try
        Parser.program Lexer.token lexbuf
      with e -> Error.readable e lexbuf
    in
    let main =
      try
        Parser.main Lexer.main lexbuf
      with e -> Error.readable e lexbuf
    in
    Ast.Program (header, processes, main)

  let print_ast ast =
    CalodsPrettyPrinter.print_program
      Format.std_formatter ast

  let check ast =
    Checker.type_check ast
end

\end{minted}
\end{frame}


\section{COVID19}
\begin{frame}{COVID-19}
  Les impacts du Covid-19 sur notre travail:
  \begin{itemize}
    \item absence de réunion physique $\Rightarrow$ communication par mails, plus complexe pour expliquer les avancements
    \item résultat semblable mais avec plus de possibilité d'éprouver le projet $\Rightarrow$ production plus robuste
  \end{itemize}
\end{frame}

\section{Conclusion}
\begin{frame}{Conclusion}
Ce que l'on a tiré du projet:
\begin{itemize}
  \item \textbf{Apprentissage}:
  \begin{itemize}
    \item compréhension des algorithmes distribués et du matériel de preuves (model checkers, schedulers, etc)
    \item maîtrise de \textit{OCaml} et mise en pratique du cours de PFA
    \item manipulation des arbres de syntaxe abstraite pour la transformation de programmes.
    \item utilisation poussée de git
  \end{itemize}
  \item \textbf{Continuité}: vérification de propriétés sur les graphes pour les comparer avec celles de \textit{Prism}
  \item \textbf{Rétrospective}:
  \begin{itemize}
    \item utilisation plus poussée des modules et interfaces de \textit{OCaml}. 
    \item amélioration de la communication au fil des mois
  \end{itemize}   
\end{itemize}
\end{frame}

\end{document}
